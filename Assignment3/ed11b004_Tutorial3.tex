% @Author: Athul Vijayan
% @Date:   2014-09-13 10:37:08
% @Last Modified by:   Athul Vijayan
% @Last Modified time: 2014-09-14 19:40:35

\documentclass[11pt,paper=a4,answers]{exam}
\usepackage{graphicx,lastpage}
\usepackage{upgreek}
\usepackage{censor}
\usepackage{amsmath,amssymb}
\usepackage{enumerate}

% For good looking code
\usepackage{listings}
\usepackage{color}
\definecolor{dkgreen}{rgb}{0,0.6,0}
\definecolor{gray}{rgb}{0.5,0.5,0.5}
\definecolor{mauve}{rgb}{0.58,0,0.82}
\definecolor{backcolour}{rgb}{0.95,0.95,0.92}

\lstset{frame=tb,
  backgroundcolor=\color{backcolour},
  language=Matlab,
  aboveskip=3mm,
  belowskip=3mm,
  showstringspaces=false,
  columns=flexible,
  basicstyle={\small\ttfamily},
  numbers=none,
  numberstyle=\tiny\color{gray},
  keywordstyle=\color{blue},
  commentstyle=\color{dkgreen},
  stringstyle=\color{mauve},
  breaklines=true,
  breakatwhitespace=true
  tabsize=3
}%============================ For good looking code

\censorruledepth=-.2ex
\censorruleheight=.1ex
\hyphenpenalty 10000
\usepackage[paperheight=10.5in,paperwidth=8.27in,bindingoffset=0in,left=0.8in,right=1in,
top=0.7in,bottom=1in,headsep=.5\baselineskip]{geometry}
\flushbottom
\usepackage[normalem]{ulem}
\renewcommand\ULthickness{2pt}   %%---> For changing thickness of underline
\setlength\ULdepth{1.5ex}%\maxdimen ---> For changing depth of underline
\renewcommand{\baselinestretch}{1}
\pagestyle{empty}

\pagestyle{headandfoot}
\headrule
\newcommand{\continuedmessage}{%
\ifcontinuation{\footnotesize Question \ContinuedQuestion\ continues\ldots}{}%
 }
\runningheader{\footnotesize ED4030}
{\footnotesize Finite element analysis}
{\footnotesize Page \thepage\ of \numpages}
\footrule
\footer{\footnotesize}
{}
{\ifincomplete{\footnotesize Question \IncompleteQuestion\ continues
on the next page\ldots}{\iflastpage{\footnotesize End of Assignment}{\footnotesize Please go        on to the next page\ldots}}}

\usepackage{cleveref}
\crefname{figure}{figure}{figures}
\crefname{question}{question}{questions}
%==============================================================
\begin{document}
\noindent
\begin{minipage}[l]{.1\textwidth}%
\noindent
\end{minipage}
\hfill
\begin{minipage}[r]{.68\textwidth}%
\begin{center}
{\large \bfseries IIT Madras \par
\Large Finite element analysis \\[2pt]
\small Assignment 3  \par}
\end{center}
\end{minipage}
\fbox{\begin{minipage}[l]{.195\textwidth}%
\noindent
{\footnotesize \today}
\end{minipage}}
\par
\noindent
\uline{ED11B004   \hfill \normalsize\emph \hfill       Athul Vijayan}
\begin{questions}
\pointformat{\boldmath\themarginpoints}
% =================Content starts ===================
\question
We have the function $\phi = 1 + sin({\pi x \over 2})$ over the range $0\leq x \leq 1$.\\
We wil aproximate $\phi$ as $\hat{\phi}$ using weighted residuals.
$$\hat{\phi} = \psi + \sum_{m=1}^M a_m N_m$$
In this problem, by inspection we decide to use $\psi = 1$ and trial functions as terms in expansion of $sin({\pi x \over 2})$.\\
As we increase the number of terms, i.e as $M \longrightarrow \infty$, the accuracy of our approximation increases.\\
\begin{align}
\sin ({\pi x \over 2}) & = ({\pi x \over 2}) - \frac{({\pi x \over 2})^3}{3!} + \frac{({\pi x \over 2})^5}{5!} - \frac{({\pi x \over 2})^7}{7!} + \cdots \\
& = \sum_{n=0}^\infty \frac{(-1)^n}{(2n+1)!}({\pi x \over 2})^{2n+1}
\end{align}
We will take first 3 terms for first aproximation and increase one term for evry next iterations upto eight terms.
$$\hat{\phi} = 1 + \sum_{m=0}^M a_m \frac{(-1)^m}{(2m+1)!}({\pi x \over 2})^{2m+1}$$
\begin{enumerate}[]
    \item \textbf{Point Collocation}\\
    In this method, we choose some points where actual and estimated value of function exactly agrees.
    we call those points collocation points. So in this solution, we need as many collocation points as the degree of polynomial of terms we include.\\
    collocation points $\mathbf{P} = [p_1, p_2, \cdots, p_M]$ be the M collocation points.\\
    $$\phi(x) -\hat{\phi}(x) = 0 \qquad \forall \quad x \in \mathbf{P}$$
    so, for every collocation point $p_i$,\\
    \begin{align}
        1 + sin({\pi p_i \over 2}) - \left(1 + \sum_{m=0}^M a_m \frac{(-1)^m}{(2m+1)!}({\pi p_i \over 2})^{2m+1}\right) &= 0 \qquad \forall \quad x \in \mathbf{P} \nonumber \\
    \end{align}
    we can write in a matrix as:
    $$[\mathbf{K}]_{M \times M} \{\mathbf{a}\}_{M \times 1} = \{\mathbf{F}\}_{M \times 1}$$
    \newpage
    where
    \begin{enumerate}[]
        \item $\{\mathbf{a}\} = \begin{bmatrix}
            a_1 & a_2 & \cdots & a_M
        \end{bmatrix}^T$
        \item $\mathbf{K}_{ij} = \frac{(-1)^j}{(2j+1)!}({\pi p_i \over 2})^{2j+1}$
        \item $\mathbf{F}_i = sin({\pi p_i \over 2})$
    \end{enumerate}
    The plot of exact and estimate function for each increase in number of terms is plotted.:
    \begin{enumerate}
        \item Iteration 1 $\longrightarrow$ first 2 terms selected from the series\\
        \centerline{\includegraphics[width=12cm]{q1ai1.png}}
        \item Iteration 2 $\longrightarrow$ first 3 terms selected from the series\\
        \centerline{\includegraphics[width=12cm]{q1ai2.png}}
        \item Iteration 3 $\longrightarrow$ first 4 terms selected from the series\\
        \centerline{\includegraphics[width=12cm]{q1ai3.png}}
        \item Iteration 4 $\longrightarrow$ first 5 terms selected from the series\\
        \centerline{\includegraphics[width=12cm]{q1ai4.png}}
        \newpage
        \item Iteration 5 $\longrightarrow$ first 6 terms selected from the series\\
        \centerline{\includegraphics[width=12cm]{q1ai5.png}}
    \end{enumerate}
    And the code for above is:\\
    \begin{lstlisting}
        clear all
        clc

        numIter = 5;                            % number of iterations
        syms x;
        initTerms = 2;                          % Number of terms in the sine expansion included for first iteration
        params = cell(numIter,1);               % unknowns
        trialFns = cell(numIter, 1);            % trial functions for each iterations
        phi = 1 + sin(pi*x/2);
        phiHat = cell(numIter, 1);

        for i=1:numIter
            % ********************** collocation method ******************.
            collocPts = transpose(linspace(0, 1, initTerms +i +1)); collocPts = collocPts(2:end-1);
            collocFvec = sin(pi*collocPts/2);
            for j=1: initTerms +i -1
                % trialFns{i}{j} = (x^j)*(x-1);
                trialFns{i}{j} = ((-1)^(j-1)*((pi*x)^(2*(j-1) + 1))) / (factorial(2*(j-1) +1)*(2^(2*(j-1)+1)));
            end

            for k=1:size(collocPts, 1)
                for j=1:size(collocPts, 1)
                    Kvec(k,j) = subs(trialFns{i}{j}, x, collocPts(k));
                end
            end
            collocParams{i} = inv(Kvec)*collocFvec;
            phiHat{i} = 1 + trialFns{i}*collocParams{i};

            figure
            phiPlot = ezplot(phi, [0, 1]);
            set(phiPlot,'LineWidth',2); set(phiPlot,'color','g'); 
            hold on;
            phiHatPlot = ezplot(phiHat{i}, [0, 1]);
            set(phiHatPlot,'LineWidth',2); set(phiHatPlot,'color','r'); set(phiHatPlot,'LineStyle','--');
            set(get(gca,'XLabel'),'String','x');
            set(get(gca,'YLabel'),'String','$$\phi, \quad \hat{\phi}$$', 'Interpreter','LaTex');
            L = legend('$$\phi = 1 + sin({\pi x \over 2})$$', '$$\hat{\phi}$$', 'Location','southeast');
            set(L,'Interpreter','LaTex');
            title(['Plot showing actual and estimated function using point collocation for iteration ', num2str(i)]);
            hold off;
        end
    \end{lstlisting}

    \item \textbf{Galerkin method}\\
    The residuals are given by $R = (\phi - \hat{\phi})$. We wish to minimize the weighted residuals given as
    $$\int \limits_x W_i(x) R(x) \mathrm{d} x = 0$$
    Where $W_i = N_i$\\
    \begin{align}
        \int \limits_x W_i(x) (\phi - \hat{\phi}) \mathrm{d} x &= 0 \qquad \forall \quad i\nonumber \\
        \int \limits_x W_i(x) \phi(x) \mathrm{d} x &= \int \limits_x W_i(x) \hat{\phi}(x) \mathrm{d} x \nonumber \\
        &= \int \limits_x N_i (1 + a_1 N_1 + a_2 N_2 + \cdots + a_M N_M) \nonumber\\
        \int \limits_x W_i(x) \phi(x) \mathrm{d} x - \int \limits_x N_i \mathrm{d} x &= [\mathbf{K}]\{\mathbf{a}\} \nonumber \\
        [\mathbf{K}]_{M \times M} \{\mathbf{a}\}_{M \times 1} &= \{\mathbf{F}\}_{M \times 1}
    \end{align}
    where
    \begin{enumerate}[]
        \item $\{\mathbf{a}\} = \begin{bmatrix}
            a_1 & a_2 & \cdots & a_M
        \end{bmatrix}^T$
        \item $\mathbf{K}_{ij} = \int \limits_x N_i N_j \mathrm{d} x$
        \item $\mathbf{F}_i = \int \limits_x N_i(\phi(x) -1) \mathrm{d}x = \int \limits_x N_i sin({\pi x \over 2}) \mathrm{d}x$
    \end{enumerate}
    and we can find $\mathbf{a}$ by $\mathbf{a} = \mathbf{K} ^{-1} \mathbf{F}$.\\
    \newpage
    code:
    \begin{lstlisting}
        clear all
        clc

        numIter = 5;                            % number of iterations
        syms x;
        initTerms = 2;                          % Number of terms in the sine expansion included for first iteration
        params = cell(numIter,1);               % unknowns
        trialFns = cell(numIter, 1);            % trial functions for each iterations
        phi = 1 + sin(pi*x/2);
        phiHat = cell(numIter, 1);

        for i=1:numIter

            % ==============galerkin method ========================
            % We start with initTerms number ofterms in sine expansion and increase terms upon iteration.
            % construct trial functions matrix.
            trialFns{i} = cell(i + initTerms -1, 1);
            for j=0: initTerms +i -2
                trialFns{i}{j+1} = ((-1)^j*((pi*x)^(2*j + 1))) / (factorial(2*j +1)*(2^(2*j+1)));
            end
            % The trialFns has trial functions $$N_i$$ now.
            % RHS vector F
            FvecIntegrand = trialFns{i} * sin(pi*x/2);     %RHS integrand
            Fvec = double(int(FvecIntegrand, x, 0, 1));    %integrate
            clear('FvecIntegrand');

            % Construct K matrix
            for j=1:size(trialFns{i},1)
                for k=1:size(trialFns{i},1)
                    KvecIntegrand(j,k) = trialFns{i}{j}*trialFns{i}{k}; % Kmatrix integrand
                end
            end
            Kvec = double(int(KvecIntegrand, x, 0, 1));     %integrate
            clear('KvecIntegrand');
            params{i} = inv(Kvec)*Fvec;                     %find unknowns
            phiHat{i} = 1;                                  %Find aproximate phi, phiHat
            collocPhiHat{i} = 1;
            for j=1:size(trialFns{i},1)
                phiHat{i} = phiHat{i} + params{i}(j)*trialFns{i}{j};
            end

            figure;
            phiPlot = ezplot(phi, [0, 1]);
            set(phiPlot,'LineWidth',2); set(phiPlot,'color','g');
            hold on;
            phiHatPlot = ezplot(phiHat{i}, [0, 1]);
            set(phiHatPlot,'LineWidth',2); set(phiHatPlot,'color','r'); set(phiHatPlot,'LineStyle','--');
            set(get(gca,'XLabel'),'String','x');
            set(get(gca,'YLabel'),'String','$$\phi, \quad \hat{\phi}$$', 'Interpreter','LaTex');
            L = legend('$$\phi = 1 + sin({\pi x \over 2})$$', '$$\hat{\phi}$$', 'Location','southeast');
            set(L,'Interpreter','LaTex');
            title(['Plot showing actual and estimated function using galerkin for iteration ', num2str(i)]);
            hold off
            clear('phiPlot', 'phiHatPlot', 'L');
        end
    \end{lstlisting}
    In the code, we have used above methodology and trial functions. Since its the expansion of $sin(x)$, we should get parametrs close to one. The plot of actual and estimated function for each iteration of trial functions are given.
    \begin{enumerate}[i.]
        \item Iteration 1 - 2 terms in the expansion.\\
        $\mathbf{a} = \begin{bmatrix}
            0.9888 & 0.8704
        \end{bmatrix}$\\
        \includegraphics[width=12cm]{q1bi1.png}
        \newpage
        \item Iteration 2 - 3 terms in the expansion.\\
        $\mathbf{a} = \begin{bmatrix}
            0.9998  &  0.9950  &  0.9089
        \end{bmatrix}$\\
        \includegraphics[width=12cm]{q1bi2.png}
        \item Iteration 3 - 4 terms in the expansion.\\
        $\mathbf{a} = \begin{bmatrix}
            1.0000  &  0.9999  &  0.9971  &  0.9297
        \end{bmatrix}$\\
        \includegraphics[width=12cm]{q1bi3.png}
        \item Iteration 4 - 5 terms in the expansion.\\
        $\mathbf{a} = \begin{bmatrix}
            1.0000  &  1.0000  &  1.0000  &  0.9982  &  0.9428
        \end{bmatrix}$\\
        \includegraphics[width=12cm]{q1bi4.png}
        \item Iteration 5 - 6 terms in the expansion.\\
        $\mathbf{a} = \begin{bmatrix}
            1.0000  &  1.0000  &  1.0000  &  1.0001  &  0.9960  &  0.9518
        \end{bmatrix}$\\
        \includegraphics[width=12cm]{q1bi5.png}
    \end{enumerate}
\end{enumerate}
\newpage
\question
Since we do not know much about the characteristics of heat flow, we cannot write down a differential equation for the flow. So with the data points we have we'll fit a function for temperature using galerkin method.\\
$$\hat{\phi} = \psi + \sum_{m=1}^M a_m N_m$$
We write our residual function as $\phi - \hat{\phi}$. So with galerkin matheod, we have our sum of weighted residuals as:
\begin{align}
    \int \limits_x W_i (\phi - \hat{\phi}) \mathrm{d x} = 0 \nonumber\\
    \int \limits_x W_i \phi \mathrm{d x} &= \int \limits_x W_i \hat{\phi} \mathrm{d x} \nonumber \\
    \int \limits_x W_i \phi \mathrm{d x} &= \int \limits_x W_i \left(\psi + \sum_{m=1}^M a_m N_m\right) \mathrm{d x} \nonumber \\
    \int \limits_x W_i (\phi - \psi) \mathrm{d x} &= \int \limits_x W_i \left(\sum_{m=1}^M a_m N_m\right) \mathrm{d x} \nonumber \\
    \int \limits_x W_i (\phi - \psi) \mathrm{d x} &= \int \limits_x W_i \left(a_1 N_1 + a_2 N_2 + a_3 N_3 + \cdots + a_m N_m\right) \mathrm{d x} \nonumber \\
\end{align}
We can vectorize the equation as 
    $$\{\mathbf{F}\}_{m \times 1} = [\mathbf{K}]_{m \times m} \{\mathbf{a}\}_{m \times 1} $$
where:
\begin{enumerate}[]
    \item $\{\mathbf{F}\}$ is found using numerical integtation
    \item $[\mathbf{K}]_{ij} = \int \limits_x W_i N_j \mathrm{d x}$
    \item $\{\mathbf{a}\} = \begin{bmatrix}
        a_1 & a_2 & a_3 & \cdots & a_m
    \end{bmatrix}$
\end{enumerate}
Trial functions chosen are such that $N_i = x^i (x - 1)$
from the condition $\phi = 20 | x=0$ and  $\phi = 30 | x=1$ gives us $\psi = 20 + 10x$.\\

Code:
\begin{lstlisting}
    for numTrialFns =4:6;                                        % do for iterations. each iteration adda a trial term
        syms x;   
        distance = transpose(linspace(0, 1, 6));                % given steps in x
        phi = [20 ; 30 ; 50 ; 65 ; 40 ; 30];                    %temperatures
        psi = 20 + 10*x;                                        % $$\psi$$ in estimate

        for i=1:numTrialFns
            trialFns(i) = (x^i)*(x-1);                          % find symbolic trial functions
        end

        for i=1:numTrialFns
            for j=1:numTrialFns
                KvecIntegrand(i,j) = trialFns(i)*trialFns(j); % construct K matrix
            end
            FvecIntegrand = double(times(subs(trialFns(i), x, distance) , phi - (subs(psi, x, distance))));
            Fvec(i) = trapz(distance, FvecIntegrand);           % do numerical integration 
        end
        Fvec = Fvec';
        Kvec = double(int(KvecIntegrand, x, 0, 1));             % integrate to get final K matrix

        params = inv(Kvec)*Fvec;                                % solved parameters
        phiHat = psi + trialFns*params;                         %Recreate symbolic estimate eqn

        % Now we plot both $$\phi$$ and $$\hat{\phi}$$.
        figure
        hold on
        phiPlot = plot(distance, phi);
        set(phiPlot,'LineStyle','o'); set(phiPlot,'color','b'); set(phiPlot,'MarkerFaceColor',[.49 1 .63]);
        phiHatPlot = ezplot(phiHat, [0, 1]);
        set(phiHatPlot,'LineWidth',2); set(phiHatPlot,'color','r');
        set(get(gca,'XLabel'),'String','x');
        set(get(gca,'YLabel'),'String','$$\phi, \quad \hat{\phi}$$', 'Interpreter','LaTex');
        L = legend('$$\phi$$', '$$\hat{\phi}$$', 'Location','southeast');
        set(L,'Interpreter','LaTex');
        title(['Plot showing actual and estimated function using galerkin for iteration ', num2str(numTrialFns - 3)]);
        clear all
        clc
    end
\end{lstlisting}
\newpage
The plots for each iteration is:\\
\begin{enumerate}[i.]
    \item Iteration 1 $\longrightarrow $ 4 terms\\
    \centerline{\includegraphics[width=10cm]{q2i1.png}}
    \item Iteration 2 $\longrightarrow $ 5 terms\\
    \centerline{\includegraphics[width=10cm]{q2i2.png}}
    \newpage
    \item Iteration 3 $\longrightarrow $ 6 terms\\
    \centerline{\includegraphics[width=10cm]{q2i3.png}}
\end{enumerate}


% subsection problem_1_part_ (end)

% =================Content ends ===================
\end{questions}
\begin{center}
\rule{.7\textwidth}{1pt}
\end{center}
\end{document} 