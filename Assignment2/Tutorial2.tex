% @Author: Athul Vijayan
% @Date:   2014-08-23 14:51:19
% @Last Modified by:   Athul Vijayan
% @Last Modified time: 2014-08-27 23:34:36

\documentclass{article}
\usepackage[utf8]{inputenc}
\usepackage[a4paper, margin=1in]{geometry}
\usepackage{amsmath}
\usepackage{graphicx}
\usepackage{listings}
\usepackage{color}

\definecolor{dkgreen}{rgb}{0,0.6,0}
\definecolor{gray}{rgb}{0.5,0.5,0.5}
\definecolor{mauve}{rgb}{0.58,0,0.82}

\lstset{frame=tb,
  language=Matlab,
  aboveskip=3mm,
  belowskip=3mm,
  showstringspaces=false,
  columns=flexible,
  basicstyle={\small\ttfamily},
  numbers=none,
  numberstyle=\tiny\color{gray},
  keywordstyle=\color{blue},
  commentstyle=\color{dkgreen},
  stringstyle=\color{mauve},
  breaklines=true,
  breakatwhitespace=true
  tabsize=3
}

\newcommand{\HRule}{\rule{\linewidth}{0.2mm}} % Defines a new command for the horizontal lines, change thickness here
\title{ED4030 Tutorial 1}
\author{Athul Vijayan}
\date{15th August 2014}
\begin{document}
\begin{titlepage}
    \center % Center everything on the page
     
    %----------------------------------------------------------------------------------------
    %   HEADING SECTIONS
    %----------------------------------------------------------------------------------------

    \textsc{\LARGE Indian Institute of Technology, Madras}\\[1.5cm] % Name of your university/college
    \textsc{\Large Finite Element Methods for Design}\\[0.5cm] % Major heading such as course name
    \textsc{\large ED4030}\\[0.5cm] % Minor heading such as course title

    %----------------------------------------------------------------------------------------
    %   TITLE SECTION
    %----------------------------------------------------------------------------------------

    \HRule \\[0.4cm]
    { \huge \bfseries Tutorial 2}\\[0.4cm] % Title of your document
    \HRule \\[1.5cm]
     
    %----------------------------------------------------------------------------------------
    %   AUTHOR SECTION
    %----------------------------------------------------------------------------------------

    \Large \emph{Submitted by:}\\
    Athul Vijayan % Your name

    ED11B004\\[8cm] % Your name
    %----------------------------------------------------------------------------------------
    %   DATE SECTION
    %----------------------------------------------------------------------------------------
    {\large \today}\\[6cm] % Date, change the \today to a set date if you want to be precise
    \vfill % Fill the rest of the page with whitespace
\end{titlepage}
\section{Problem 3}
Notations used are:
\begin{enumerate}
    \item Temperature at grid point $(i, j)$ (where i along horizontal and j along vertical) is $\phi_{ij}$
    \item Derivatives are represented by $\frac{\partial \phi}{\partial x} \rightarrow \phi^{i}_{i,j}$, $\frac{\partial \phi}{\partial y} \rightarrow \phi^{j}_{i,j}$, $\frac{\partial ^2 \phi}{\partial x^2} \rightarrow \phi^{ii}_{i,j}$, $\frac{\partial ^2 \phi}{\partial xy} \rightarrow \phi^{ij}_{i,j}$, $\frac{\partial ^2 \phi}{\partial y^2} \rightarrow \phi^{jj}_{i,j}$
\end{enumerate}
The given grid for studying the heat distribution is:\\
\centerline{\includegraphics[width=5cm]{grid.png}}
With Length $=6$ and Width $=8$.\\
For finite elements with $N\Delta x = 6$ and $M\Delta y = 8$, 
Since internal heat generation is Zero, The governing equation is for heat flow is:\\
\centerline{$k\left(\frac{\partial ^2 \phi}{\partial ^2 x} + \frac{\partial ^2 \phi}{\partial ^2 y}\right) = 0$}
Where the boundary conditions for the grid are:\\
For $\Delta x =\{1, 1/2,1/4..., \}$ and $\Delta y =\{1, 1/2,1/4..., \}$\\
\begin{align}
      \phi_{\frac{2}{\Delta x}, \frac{j}{\Delta y}} &= 100 \qquad \text{for $j = \{2, 3, 4, 5, 6\}$} \\
      \phi_{\frac{4}{\Delta x}, \frac{j}{\Delta y}} &= 100 \qquad \text{for $j = \{2, 3, 4, 5, 6\}$} \\
      \phi_{\frac{3}{\Delta x}, \frac{2}{\Delta y}} &= 100 \\
      \phi_{\frac{3}{\Delta x}, \frac{6}{\Delta y}} &= 100 \\
      k\frac{\partial \phi}{\partial x} &= -\alpha \phi \qquad \text{at $i=\{0, \frac{n}{\Delta x}\}$}\\ 
      k\frac{\partial \phi}{\partial y} &= -\alpha \phi \qquad \text{at $j=\{0, \frac{m}{\Delta y}\}$}
\end{align}
With finite elements aproximation, the governing equation can be written as:\\
\begin{align}
    \frac{\phi_{i+1, j} - 2\phi_{i, j} + \phi_{i-1, j}}{\Delta x^2} + \frac{\phi_{i, j+1} - 2\phi_{i, j} + \phi_{i, j-1}}{\Delta y^2} = 0 \nonumber
\end{align}
with $\Delta x = \Delta y = h$\\
\begin{align}
    \phi_{i+1, j} - 4\phi_{i, j} + \phi_{i-1, j} + \phi_{i, j+1} + \phi_{i, j-1}= 0 \nonumber
\end{align}
Now we need to make the recursive equation into a matrix equation in the form of $AX=B$. For coding simplicity, we initially consider matrix $\mathbf{\Phi}$ as an (n-1)(m-1) x 1 matrix as:\\
\centerline{$\mathbf{\Phi} = 
\begin{bmatrix}
    \phi_{1,1} & \phi_{2,1} & \phi_{3,1} &\cdots & \phi_{n-1,1} & \phi_{1,2} & \phi_{2,2} &\phi_{n-1, 2} &\cdots & \phi_{n-1, m-1} 
\end{bmatrix}^T$}\\
\newpage
For now, we does not care about boundary condition; we will do that after a while.\\
And we define matric $B_{n-1Xn-1}$ as:\\
\centerline{$\bf{B}_{n-1Xn-1} = 
\begin{pmatrix}
    -4 & 1 & 0 & 0 & \cdots & 0 \\
    1 & -4 & 1 & 0 & \cdots & 0 \\
    0 & 1 & -4 & 1 & \cdots & 0 \\
    \vdots  & \vdots  & \vdots & \vdots & \ddots & \vdots \\
    0& 0 & 0 & \cdots & 1 & -4 
\end{pmatrix}$}\\
And $I_{n-1Xn-1}$ as:\\
\centerline{$\bf{I}_{n-1Xn-1} = 
\begin{pmatrix}
    1 & 0 & 0 &  \cdots & 0 \\
    0 & 1 & 0 &  \cdots & 0 \\
    0 & 0 & 1 &  \cdots & 0 \\
    \vdots  &  \vdots & \vdots & \ddots & \vdots \\
    0& 0 & 0 & \cdots  & 1 
\end{pmatrix}$}\\
and a zero matrix is defined as:\\
\centerline{$\bf{0}_{mXm} = 
\begin{pmatrix}
    0 & 0 & 0 &  \cdots & 0 \\
    0 & 0 & 0 &  \cdots & 0 \\
    0 & 0 & 0 &  \cdots & 0 \\
    \vdots  &  \vdots & \vdots & \ddots & \vdots \\
    0& 0 & 0 & \cdots  & 0 
\end{pmatrix}$}\\
The governing equation with zero initial conditions (all border points are at zero temperature) applied to a rectangular sheet in 2D can be written as:\\
\centerline{$\bf{A\Phi = F}$}.\\
where A is defined as:\\
\centerline{$\bf{A}_{(n-1)(m-1) x (n-1)(m-1)} = 
\begin{pmatrix}
    \bf{B} & \bf{I} & \bf{0} & \bf{0} & \cdots & \bf{0} \\
    \bf{I} & \bf{B} & \bf{I} & \bf{0} & \cdots & \bf{0} \\
    \bf{0} & \bf{I} & \bf{B} & \bf{I} &\cdots & \bf{0} \\
    \vdots  &  \vdots & \vdots & \vdots& \ddots & \vdots \\
    \bf{0} & \bf{0} & \bf{0} & \cdots  &  \bf{I} & \bf{B}
\end{pmatrix}$}\\
and $\bf{F}$ is\\
\centerline{$\bf{A_{(n-1)(m-1) \times 1}} = \begin{bmatrix}
    0 & 0 & 0 & 0 & \cdots & 0
\end{bmatrix}^T$}
To work with the hollow part, we can use the symmetry of the given plate. If we divide the plate into four as shown below, we just need to add another boundary condition for the newly made sides.\\
\centerline{\includegraphics[width=5cm]{gridsection.png}}\\
Since the shape is not rectangular, we need to modify the $\bf{A}$ matrix for this.\\
Say the number of elements inside one block in the figure is $\frac{1}{\Delta x}$.
\begin{enumerate}
    \item Length along x (i) direction is more on the bottom part, which implies the dimension of square matrix $\bf{B}$ will be more at the bottom.
    \item Dimension of $\bf{B}$ and $\bf{I}$ on $j \leq \frac{2}{\Delta y}$ is $(\frac{3}{\Delta x}-1) \times (\frac{3}{\Delta x}-1)$
    \item Length along x (i) direction is less on the top part, which implies the dimension of square matrix $\bf{B}$ will be less at the top.
    \item Dimension of $\bf{B}$ and $\bf{I}$ on $j > \frac{2}{\Delta y}$ is $(\frac{2}{\Delta x}-1) \times (\frac{2}{\Delta x}-1)$
\end{enumerate}
Finally, matrices for our problem with zero boundary conditions are:\\
\centerline{$\mathbf{\Phi} = 
\begin{bmatrix}
    \phi_{1,1} & \phi_{2,1} & \cdots & \phi_{(\frac{3}{\Delta x}-1),1} & \phi_{1,2} &\cdots&\phi_{(\frac{3}{\Delta x}-1), 2} &\cdots & \phi_{(\frac{3}{\Delta x}-1), (\frac{2}{\Delta y}-1)} & \cdots
\end{bmatrix}^T$}\\
matrix $\bf{\Phi}$ is all the points inside the mesh (excluding all boundary points) as shown in the order.\\
For e.g. for $\Delta x = \Delta y = 1$,\\
\centerline{$\mathbf{\Phi} = 
\begin{bmatrix}
    \phi_{1,1} & \phi_{2,1} & \phi_{1,2} & \phi_{1,3}
\end{bmatrix}^T$}\\
matrix $\bf{F}$ is zero matrix.\\
So for our probem, with zero boundary conditions, the equation in matrix form can be formed by following code.\\
\begin{enumerate}
    \item Step 1: Make the A matrix for bottom rectangular part and apply Derivative BC at $i=0$ and $i=n-1$\\
    \begin{lstlisting}
        clear all
        numGrids = 5;                       % number of different grids over which solution is sought

        xArray = cell(numGrids,1);          % array storing the x values at grid locations
        yArray = cell(numGrids,1);

        phiFD = cell(numGrids,1);           % solution vectors
        midValsFD = zeros(numGrids,1);      % an array storing all phi values at x , FD
        stepVals = zeros(numGrids,1);       % array storing delta x values for different grids
        phi_0 = 0;                          % phi b.c. at x = 0
        i=2;
        % for i = 1:numGrids
            % =========== First we produce the matrix A for the bottom rectangular part. ==============
            xNumFirstGridElems = 3;
            yNumFirstGridElems = 2; 
            xNumElem = xNumFirstGridElems*2^(i-1);      % x number of elements
            yNumElem = yNumFirstGridElems*2^(i-1);      % y number of elements

            deltaX = 1/yNumElem;

            % make B matrix., so that we can just pile this to form A matrix
            BdiagVec = -4*ones(xNumElem-1,1);           % exclude two boundaries
            BdiagVec(1) = -3+deltaX;                    %add derivative boundary condition, at all i = 1 for bottom part
            BdiagVec(xNumElem-1) = -3+deltaX;           %add derivative boundary condition, at all i = n-1 for bottom part
            if (xNumElem-1 > 1)
                B_bottom = gallery('tridiag',ones(xNumElem -2, 1),BdiagVec,ones(xNumElem -2, 1)); % matrix of coefficients
            else
                B_bottom = BdiagVec                     % case for $$\Delta x =1$$
            end

            A = B_bottom;                               %Initialize A matrix
            for k=1:yNumElem-2
                A = blkdiag(A, B_bottom);               %populate A matrix with B matrix as diagonal
            end
            % Now loop through the A matrix to add I matrix wherever necessary
            for k= 1: (xNumElem - 1)*(yNumElem-1)
                if (k + xNumElem -1 <= size(A, 2))      % leave the boundary, i=n and j=m
                    A(k, k + xNumElem -1) = 1;
                end
                if (k > (xNumElem - 1))                 % leave the boundary, i=0 and j=0
                    A(k, k -( xNumElem -1)) = 1;
                end
            end
            % Now the vector A has the coefficients of $$\phi$$s inside the bottom rectangle.
                % ====================================================================
    \end{lstlisting}
    \item Step 2: populate A matrix to include equations for top rectangular part and apply Derivative BC at $i=0$ and $i=n-1$
    \begin{lstlisting}
        % Now we add to this matrix, the coefficients for the equations of top rectangular part
        % having different dimensions. So this will result in change of dimesion of B matrix and I matrix

        % =========== then we produce the matrix A for the bottom rectangular part. ==============
        xNumFirstGridElems = 2;
        yNumFirstGridElems = 2;
        yNumElemOld = yNumElem;
        xNumElemOld = xNumElem;
        xNumElem = xNumFirstGridElems*2^(i-1);      % number of x elements
        yNumElem = yNumFirstGridElems*2^(i-1);      % number of y elements

        BdiagVec = -4*ones(xNumElem-1,1);           % exclude two boundaries
        BdiagVec(1) = -3+deltaX;                    %add derivative boundary condition, at all i = 1 for bottom part
        if (xNumElem-1 > 1)
            B_top = gallery('tridiag',ones(xNumElem -2, 1),BdiagVec,ones(xNumElem -2, 1)); % matrix of coefficients
        else
            B_top = BdiagVec;
        end

        for k=1: yNumElem
            A = blkdiag(A, B_top);                  %populate A matrix with B matrix as diagonal
        end
        % Now loop through the A matrix to add I matrix wherever necessary
        for k= (yNumElemOld-1)*(xNumElemOld-1) + 1: (yNumElemOld-1)*(xNumElemOld-1) + (xNumElem - 1)*(yNumElem)
            if (k + xNumElem -1 <= size(A, 2))   
                A(k, k + xNumElem -1) = 1;
            end
            if (k > (yNumElemOld-1)*(xNumElemOld-1) + (xNumElem - 1))
                A(k, k -( xNumElem -1)) = 1;
            end
        end
            % ====================================================================
    \end{lstlisting}
    \item Step 3: Even though we added two matrices, we haven't told them about the boundary they are sharing in the common area. if we don't do that, it will take zero boundaries (current situation) which will be false. So we add these common boundary to matrix.\\
    \begin{lstlisting}
        % i.e. in the sheet, we haven't considered the fact that there is no boundary 
        % between this top and bottom plate upto a point. right now the matrix assumes, zero boundary condition.
        % to fix this we can link by following code
        for k= (xNumElemOld - 1)*(yNumElemOld-2): (xNumElemOld - 1)*(yNumElemOld-2) + (xNumElem - 1)
            if k > 0
                A(k, k + xNumElemOld -1) = 1;
            end
        end
        for k= (xNumElemOld - 1)*(yNumElemOld-1) + 1: (xNumElemOld - 1)*(yNumElemOld-1) + xNumElem -1
            A(k, k - (xNumElemOld -1)) = 1;
        end
        % Now the matrix is complete for the given plate with boundary condition as zero along boundaries.
    \end{lstlisting}
\end{enumerate}
Now we apply first boundary condition.\\
for $k=1$ and $\alpha = 1$.
\begin{enumerate}
    \item $\phi_{n-1} - (\Delta x +1)\phi_{n} = 0 \qquad$ at outer boundaries $i=1$ and $j=1$; using backward difference.
    \item $\phi_{n+1} - (\Delta x +1)\phi_{n} = 0 \qquad$ at outer boundaries $i=n-1$ and $j=m-1$; using forward difference.
\end{enumerate}
This can be included by adding $(1 + \Delta x)$ to $-4$ cells in the matrix where the boundary conditions apply.\\
in code we can loop for these entries and add $(1 + \Delta x)$ to -4's:\\
\begin{enumerate}
    \item At boundaries where $y = 1$:\\
    \begin{lstlisting}
           % We now add boundary conditions one by one

            % Now add derivative boundary condition at j=1
            y = 1;
            for index = y:y + (xNumElemOld-2)
                id = find(A(index, :) == -4);
                A(index, id) = -3 + deltaX;
            end
    \end{lstlisting}
    \item At boundaries where $y = m-1$:\\
    \begin{lstlisting}
        % Now add derivative boundary condition at j=m-1
        y = (xNumElemOld - 1)*(yNumElemOld-1) + (xNumElem -1)*(yNumElem-1) + 1;
        for index = y :y + (xNumElem-2)
            id = find(A(index, :) == -4);
            A(index, id) = -3 + deltaX;
        end
    \end{lstlisting}
    \item Now, always three corners will have terms coming from both boundaries conditions e.g. i=1 and j=1 or i=1, j=m-1 etc.. to solve that:\\
    \begin{lstlisting}
        % 3 corner points, where two boundary terms will come
        A(1, 1) = A(1, 1) + (1+deltaX);
        A(xNumElemOld -1, xNumElemOld -1) = A(xNumElemOld -1, xNumElemOld -1) + (1+deltaX);
        A(size(A,1)-(xNumElem-2), size(A,1)-(xNumElem-2)) = A(size(A,1)-(xNumElem-2), size(A,1)-(xNumElem-2)) + (1+deltaX);
        clear('y', 'id', 'index');
    \end{lstlisting}
\end{enumerate}
\newpage
Now second boundary condition, the temperatures of inner walls, can be included by adding constants to vector $\bf{F}$ on the right hand side.\\
new $\bf{F}$ is created by
\begin{lstlisting}
    % initialize F
    F = zeros(size(A,1),1);
    % assign -100 to points corresponding to ones near inner boundary
    for k=(yNumElemOld-2)*(xNumElemOld-1)+ xNumElem:(yNumElemOld-2)*(xNumElemOld-1)+xNumElemOld-1
        F(k) = -100;
    end

    for y=1:yNumElem
        id = (yNumElemOld-1)*(xNumElemOld-1)+ y*(xNumElem-1);
        F(id) = -100;
    end

\end{lstlisting}
Invert the matrix to find unknown temperatures\\
\begin{lstlisting}
    
    % Now we have all the required matrices. we just need to invert A matrix to get temperature distribution
    phiVector{i} = inv(A)*F;
\end{lstlisting}
\begin{enumerate}
    \item Now we need to convert that row matrix into a 2D matrix for contour potting\\
    \begin{lstlisting}
        % Now we need to convert that row matrix into a 2D matrix for contour potting.
        % It is done by
        for y=1:size(phi{i}, 1)
            for x=1:size(phi{i}, 2)
                % update interior points
                if ((y < yNumElemOld + 1) && (y >1))
                    phi{i}(y, 2: xNumElemOld) = phiVector{i}((y-1)*(xNumElemOld-1) +1: y*(xNumElemOld-1) );
                end

                % update internal boundary
                if (y > yNumElemOld)
                    phi{i}(y, 2: xNumElem) = phiVector{i}((y-1)*(xNumElem-1) +1: y*(xNumElem-1) );
                    if (x == xNumElem +1)
                        phi{i}(y, x) = -100;
                    end
                end
                % update internal boundary
                if (y == yNumElemOld)
                    phi{i}(y, xNumElem+1: end) = -100;
                end
            end
        end

    \end{lstlisting}
    \item update outer boundaries\\
    \begin{lstlisting}
        % update outer boundaries
        for y=1:size(phi{i}, 1)
            phi{i}(y, 1) = (1+deltaX)* phi{i}(y, 2);
            if y<yNumElemOld
                phi{i}(y, xNumElemOld + 1) = (1+deltaX)* phi{i}(y, xNumElemOld);
            end
        end

        % update outer boundaries
        for x=1:size(phi{i}, 2)
            phi{i}(1, x) = (1+deltaX)* phi{i}(2, x);
        end
    \end{lstlisting}
    \item Since we divided sheet to 4, remake that full matrix\\
    \begin{lstlisting}
        phi{i} = horzcat(phi{i}, flip(phi{i}(:, 2:end), 2));
        phi{i} = vertcat(phi{i}, flip(phi{i}(2:end, :)));
    \end{lstlisting}
    \item Plot contour
    \begin{lstlisting}
        % grid points
        xArray{i} = 0:2*deltaX:6;
        yArray{i} = 0:2*deltaX:8;

        % plot the contour
        figure;
        [X,Y] = meshgrid(xArray{i},yArray{i});
        contour(X, Y, phi{i}, 30);
    \end{lstlisting}
    \centerline{\includegraphics[width=10cm]{contour.png}}

\end{enumerate}


    
\end{document}